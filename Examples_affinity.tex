\pagebreak
\chapter{The \code{proc\_bind} Clause}
\label{chap:affinity}

The following examples demonstrate how to use the \code{proc\_bind} clause to 
control the thread binding for a team of threads in a \code{parallel} region. 
The machine architecture is depicted in the figure below. It consists of two sockets, 
each equipped with a quad-core processor and configured to execute two hardware 
threads simultaneously on each core. These examples assume a contiguous core numbering 
starting from 0, such that the hardware threads 0,1 form the first physical core.

\ifpdf
%\begin{figure}[htbp]
\centerline{\includegraphics[width=3.8in,keepaspectratio=true]%
{figs/proc_bind_fig.pdf}}
%\end{figure}
\fi

The following equivalent place list declarations consist of eight places (which 
we designate as p0 to p7):

\code{OMP\_PLACES=\texttt{"}\{0,1\},\{2,3\},\{4,5\},\{6,7\},\{8,9\},\{10,11\},\{12,13\},\{14,15\}\texttt{"}}

or

\code{OMP\_PLACES=\texttt{"}\{0:2\}:8:2\texttt{"}}

\section{Spread Affinity Policy}

The following example shows the result of the \code{spread} affinity policy on 
the partition list when the number of threads is less than or equal to the number 
of places in the parent's place partition, for the machine architecture depicted 
above. Note that the threads are bound to the first place of each subpartition.

\cexample{affinity}{1c}

\fexample{affinity}{1f}

It is unspecified on which place the master thread is initially started. If the 
master thread is initially started on p0, the following placement of threads will 
be applied in the parallel region:

\begin{compactitem}
\item thread 0 executes on p0 with the place partition p0,p1

\item thread 1 executes on p2 with the place partition p2,p3

\item thread 2 executes on p4 with the place partition p4,p5

\item thread 3 executes on p6 with the place partition p6,p7
\end{compactitem}


If the master thread would initially be started on p2, the placement of threads 
and distribution of the place partition would be as follows:

\begin{compactitem}
\item thread 0 executes on p2 with the place partition p2,p3

\item thread 1 executes on p4 with the place partition p4,p5

\item thread 2 executes on p6 with the place partition p6,p7

\item thread 3 executes on p0 with the place partition p0,p1
\end{compactitem}

The following example illustrates the \code{spread} thread affinity policy when 
the number of threads is greater than the number of places in the parent's place 
partition.

Let \plc{T} be the number of threads in the team, and \plc{P} be the number of places in the 
parent's place partition. The first \plc{T/P} threads of the team (including the master 
thread) execute on the parent's place. The next \plc{T/P} threads execute on the next 
place in the place partition, and so on, with wrap around. 

\cexample{affinity}{2c}

\fexample{affinity}{2f}

It is unspecified on which place the master thread is initially started. If the 
master thread is initially started on p0, the following placement of threads will 
be applied in the parallel region:

\begin{compactitem}
\item threads 0,1 execute on p0 with the place partition p0

\item threads 2,3 execute on p1 with the place partition p1

\item threads 4,5 execute on p2 with the place partition p2

\item threads 6,7 execute on p3 with the place partition p3

\item threads 8,9 execute on p4 with the place partition p4

\item threads 10,11 execute on p5 with the place partition p5

\item threads 12,13 execute on p6 with the place partition p6

\item threads 14,15 execute on p7 with the place partition p7
\end{compactitem}

If the master thread would initially be started on p2, the placement of threads 
and distribution of the place partition would be as follows:

\begin{compactitem}
\item threads 0,1 execute on p2 with the place partition p2

\item threads 2,3 execute on p3 with the place partition p3

\item threads 4,5 execute on p4 with the place partition p4

\item threads 6,7 execute on p5 with the place partition p5

\item threads 8,9 execute on p6 with the place partition p6

\item threads 10,11 execute on p7 with the place partition p7

\item threads 12,13 execute on p0 with the place partition p0

\item threads 14,15 execute on p1 with the place partition p1
\end{compactitem}

\section{Close Affinity Policy}

The following example shows the result of the \code{close} affinity policy on 
the partition list when the number of threads is less than or equal to the number 
of places in parent's place partition, for the machine architecture depicted above. 
The place partition is not changed by the \code{close} policy.

\cexample{affinity}{3c}

\fexample{affinity}{3f}

It is unspecified on which place the master thread is initially started. If the 
master thread is initially started on p0, the following placement of threads will 
be applied in the \code{parallel} region:

\begin{compactitem}
\item thread 0 executes on p0 with the place partition p0-p7

\item thread 1 executes on p1 with the place partition p0-p7

\item thread 2 executes on p2 with the place partition p0-p7

\item thread 3 executes on p3 with the place partition p0-p7
\end{compactitem}

If the master thread would initially be started on p2, the placement of threads 
and distribution of the place partition would be as follows:

\begin{compactitem}
\item thread 0 executes on p2 with the place partition p0-p7

\item thread 1 executes on p3 with the place partition p0-p7

\item thread 2 executes on p4 with the place partition p0-p7

\item thread 3 executes on p5 with the place partition p0-p7
\end{compactitem}

The following example illustrates the \code{close} thread affinity policy when 
the number of threads is greater than the number of places in the parent's place 
partition.

Let \plc{T} be the number of threads in the team, and \plc{P} be the number of places in the 
parent's place partition. The first \plc{T/P} threads of the team (including the master 
thread) execute on the parent's place. The next \plc{T/P} threads execute on the next 
place in the place partition, and so on, with wrap around. The place partition 
is not changed by the \code{close} policy.

\cexample{affinity}{4c}

\fexample{affinity}{4f}

It is unspecified on which place the master thread is initially started. If the 
master thread is initially running on p0, the following placement of threads will 
be applied in the parallel region:

\begin{compactitem}
\item threads 0,1 execute on p0 with the place partition p0-p7

\item threads 2,3 execute on p1 with the place partition p0-p7

\item threads 4,5 execute on p2 with the place partition p0-p7

\item threads 6,7 execute on p3 with the place partition p0-p7

\item threads 8,9 execute on p4 with the place partition p0-p7

\item threads 10,11 execute on p5 with the place partition p0-p7

\item threads 12,13 execute on p6 with the place partition p0-p7

\item threads 14,15 execute on p7 with the place partition p0-p7
\end{compactitem}

If the master thread would initially be started on p2, the placement of threads 
and distribution of the place partition would be as follows:

\begin{compactitem}
\item threads 0,1 execute on p2 with the place partition p0-p7

\item threads 2,3 execute on p3 with the place partition p0-p7

\item threads 4,5 execute on p4 with the place partition p0-p7

\item threads 6,7 execute on p5 with the place partition p0-p7

\item threads 8,9 execute on p6 with the place partition p0-p7

\item threads 10,11 execute on p7 with the place partition p0-p7

\item threads 12,13 execute on p0 with the place partition p0-p7

\item threads 14,15 execute on p1 with the place partition p0-p7
\end{compactitem}

\section{Master Affinity Policy}

The following example shows the result of the \code{master} affinity policy on 
the partition list for the machine architecture depicted above. The place partition 
is not changed by the master policy.

\cexample{affinity}{5c}

\fexample{affinity}{5f}

It is unspecified on which place the master thread is initially started. If the 
master thread is initially running on p0, the following placement of threads will 
be applied in the parallel region:

\begin{compactitem}
\item threads 0-3 execute on p0 with the place partition p0-p7
\end{compactitem}

If the master thread would initially be started on p2, the placement of threads 
and distribution of the place partition would be as follows:

\begin{compactitem}
\item threads 0-3 execute on p2 with the place partition p0-p7
\end{compactitem}


