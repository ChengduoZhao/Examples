\pagebreak
\chapter{Asynchronous Execution of a \code{target} Region Using Tasks}
\label{chap:async_target}

The following example shows how the \code{task} and \code{target} constructs 
are used to execute multiple \code{target} regions asynchronously. The task that 
encounters the \code{task} construct generates an explicit task that contains 
a \code{target} region. The thread executing the explicit task encounters a task 
scheduling point while waiting for the execution of the \code{target} region 
to complete, allowing the thread to switch back to the execution of the encountering 
task or one of the previously generated explicit tasks.

\cexample{async_target}{1c}

The Fortran version has an interface block that contains the \code{declare} \code{target}. 
An identical statement exists in the function declaration (not shown here).

\fexample{async_target}{1f}

The following example shows how the \code{task} and \code{target} constructs 
are used to execute multiple \code{target} regions asynchronously. The task dependence 
ensures that the storage is allocated and initialized on the device before it is 
accessed.

\cexample{async_target}{2c}

The Fortran example below is similar to the C version above. Instead of pointers, though, it uses
the convenience of Fortran allocatable arrays on the device. An allocatable array has the
same behavior in a \code{map} clause as a C pointer, in this case.

If there is no shape specified for an allocatable array in a \code{map} clause, only the array descriptor
(also called a dope vector) is mapped. That is, device space is created for the descriptor, and it
is initially populated with host values. In this case, the \plc{v1} and \plc{v2} arrays will be in a
non-associated state on the device. When space for \plc{v1} and \plc{v2} is allocated on the device
the addresses to the space will be included in their descriptors.

At the end of the first \code{target} region, the descriptor (of an unshaped specification of an allocatable
array in a \code{map} clause) is returned with the raw device address of the allocated space.
The content of the array is not returned. In the example the data in arrays \plc{v1} and \plc{v2}
are not returned. In the second \code{target} directive, the \plc{v1} and \plc{v2} descriptors are
re-created on the device with the descriptive information; and references to the
vectors point to the correct local storage, of the space that was not freed in the first \code{target}
directive.  At the end of the second \code{target} region, the data in array \plc{p} is copied back
to the host since \plc{p} is not an allocatable array.

\fexample{async_target}{2f}


