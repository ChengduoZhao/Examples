\pagebreak
\chapter{Fortran Restrictions on \code{shared} and \code{private} Clauses with Common Blocks}
\fortranspecificstart
\label{chap:fort_sp_common}

When a named common block is specified in a \code{private}, \code{firstprivate}, 
or \code{lastprivate} clause of a construct, none of its members may be declared 
in another data-sharing attribute clause on that construct. The following examples 
illustrate this point. 

The following example is conforming:

\fnexample{fort_sp_common}{1f}

The following example is also conforming:

\fnexample{fort_sp_common}{2f}
% blue line floater at top of this page for "Fortran, cont."
\begin{figure}[t!]
\linewitharrows{-1}{dashed}{Fortran (cont.)}{8em}
\end{figure}

The following example is conforming:

\fnexample{fort_sp_common}{3f}

The following example is non-conforming because \code{x} is a constituent element 
of \code{c}:

\fnexample{fort_sp_common}{4f}

The following example is non-conforming because a common block may not be declared 
both shared and private:

\fnexample{fort_sp_common}{5f}
\fortranspecificend


