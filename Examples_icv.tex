\pagebreak
\chapter{Internal Control Variables (ICVs)}
\label{chap:icv}

According to Section 2.3 of the OpenMP 4.0 specification, an OpenMP implementation must act as if there are ICVs that control 
the behavior of the program.  This example illustrates two ICVs, \plc{nthreads-var} 
and \plc{max-active-levels-var}. The \plc{nthreads-var} ICV controls the 
number of threads requested for encountered parallel regions; there is one copy 
of this ICV per task. The \plc{max-active-levels-var} ICV controls the maximum 
number of nested active parallel regions; there is one copy of this ICV for the 
whole program.

In the following example, the \plc{nest-var}, \plc{max-active-levels-var}, 
\plc{dyn-var}, and \plc{nthreads-var} ICVs are modified through calls to 
the runtime library routines \code{omp\_set\_nested},\\ \code{omp\_set\_max\_active\_levels},\code{ 
omp\_set\_dynamic}, and \code{omp\_set\_num\_threads} respectively. These ICVs 
affect the operation of \code{parallel} regions. Each implicit task generated 
by a \code{parallel} region has its own copy of the \plc{nest-var, dyn-var}, 
and \plc{nthreads-var} ICVs.

In the following example, the new value of \plc{nthreads-var} applies only to 
the implicit tasks that execute the call to \code{omp\_set\_num\_threads}. There 
is one copy of the \plc{max-active-levels-var} ICV for the whole program and 
its value is the same for all tasks. This example assumes that nested parallelism 
is supported.

The outer \code{parallel} region creates a team of two threads; each of the threads 
will execute one of the two implicit tasks generated by the outer \code{parallel} 
region.

Each implicit task generated by the outer \code{parallel} region calls \code{omp\_set\_num\_threads(3)}, 
assigning the value 3 to its respective copy of \plc{nthreads-var}. Then each 
implicit task encounters an inner \code{parallel} region that creates a team 
of three threads; each of the threads will execute one of the three implicit tasks 
generated by that inner \code{parallel} region.

Since the outer \code{parallel} region is executed by 2 threads, and the inner 
by 3, there will be a total of 6 implicit tasks generated by the two inner \code{parallel} 
regions.

Each implicit task generated by an inner \code{parallel} region will execute 
the call to\\ \code{omp\_set\_num\_threads(4)}, assigning the value 4 to its respective 
copy of \plc{nthreads-var}.

The print statement in the outer \code{parallel} region is executed by only one 
of the threads in the team. So it will be executed only once.

The print statement in an inner \code{parallel} region is also executed by only 
one of the threads in the team. Since we have a total of two inner \code{parallel} 
regions, the print statement will be executed twice -- once per inner \code{parallel} 
region.

\cexample{icv}{1c}

\fexample{icv}{1f}

