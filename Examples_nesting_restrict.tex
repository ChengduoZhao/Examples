\pagebreak
\chapter{Restrictions on Nesting of Regions}
\label{chap:nesting_restrict}

The examples in this section illustrate the region nesting rules. 

The following example is non-conforming because the inner and outer loop regions 
are closely nested:

\cexample{nesting_restrict}{1c}

\fexample{nesting_restrict}{1f}

The following orphaned version of the preceding example is also non-conforming:

\cexample{nesting_restrict}{2c}

\fexample{nesting_restrict}{2f}

The following example is non-conforming because the loop and \code{single} regions 
are closely nested:

\cexample{nesting_restrict}{3c}

\fexample{nesting_restrict}{3f}

The following example is non-conforming because a \code{barrier} region cannot 
be closely nested inside a loop region:

\cexample{nesting_restrict}{4c}

\fexample{nesting_restrict}{4f}

The following example is non-conforming because the \code{barrier} region cannot 
be closely nested inside the \code{critical} region. If this were permitted, 
it would result in deadlock due to the fact that only one thread at a time can 
enter the \code{critical} region:

\cexample{nesting_restrict}{5c}

\fexample{nesting_restrict}{5f}

The following example is non-conforming because the \code{barrier} region cannot 
be closely nested inside the \code{single} region. If this were permitted, it 
would result in deadlock due to the fact that only one thread executes the \code{single} 
region:

\cexample{nesting_restrict}{6c}

\fexample{nesting_restrict}{6f}


